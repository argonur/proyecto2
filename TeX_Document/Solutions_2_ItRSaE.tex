% This document is under construction

\documentclass{article}
\usepackage[utf8]{inputenc}
\usepackage{hyperref}
\usepackage{graphicx}
\usepackage{multirow}
\usepackage{mhchem}
\usepackage{gensymb}

\begin{document}

\title{Solution to the exercises of the book ``Introduction to Rocket Science and Engineering''}
\author{Arturo Gonzalez\\
	\texttt{\href{mailto:arturo.gonzalez@argonur.com}{arturo.gonzalez@argonur.com}}}
\date{\today}
\maketitle

\begin{abstract}
This document contains the solution to the exercises of each chapter of the book ``Introduction to Rocket Science and Engineering'' by Travis S. Taylor. Solving the exercises is a project proposed in the blog \url{https://www.argonur.com}.\\

To visit the website of the project: \\
\url{https://argonur.com/solucion-introduction-to-rocket-science-and-engineering/}.\\

If you find errors in the document, want to participate solving the exercises of the book, etc., please send an email to \href{mailto:arturo.gonzalez@argonur.com}{arturo.gonzalez@argonur.com}
\end{abstract}

\cleardoublepage
%%%%%%%%%%%%%%%%%%%%%%%%%%%%%
%					Chapter 1											 %
%%%%%%%%%%%%%%%%%%%%%%%%%%%%%

\section{Exercises of Chapter 1}

\begin{enumerate}
	\item {\bf Discuss the relevance of the \textit{aelopile} to rocket science and why it was considered the first demonstration of the principle of rocketry.}\\

	\item {\bf What are the main components of the gun powder?}\\
	
	\item {\bf What was \textit{Principia} and why is it relevant to rocket science?}\\
	
	\item {\bf Why were William Hale’s rockets “better” than William Congreve’s?}\\
	
	\item {\bf Compare and contrast the contributions to the development of rocketry by Konstantin Tsiolkovsky and Robert Goddard. Which one could be considered the “father of rocket science” and which one the “father of rocket engineering”?}\\
	
	\item {\bf Who was known as the Chief Designer and why?}\\
	
	\item {\bf Who was the Chief Designer’s counterpart in the American space program?}\\
	
	\item {\bf What is the oldest spacecraft still in orbit?}\\
	
	\item {\bf What is UDMH? What is it used for? What is NTO?}\\
	
	\item {\bf Draw a simple liquid fuel rocket and label all the major subcomponents.}\\
	
\end{enumerate}


\cleardoublepage
%%%%%%%%%%%%%%%%%%%%%%%%%%%%%
%					Chapter 2											 %
%%%%%%%%%%%%%%%%%%%%%%%%%%%%%

%\section{Exercises of Chapter 2}

\begin{enumerate}

	\item {\bf Discuss the dichotomy of rocket science in the modern era.}\\
	
The first dichotomy starts after WWI, with the Treaty of Versailles that prohibited Germany to develop long-range artillery and the NAZI Germany developed the V2 (\textit{Vergeltungswaffe 2}), that was a missile capable of reaching an objective 300 km away from the launch place. The ''success'' of this lethal weapon, led the governments of the Soviet Union and USA to start developing their own missiles.

The second dichotomy happens when the Soviets where the firsts to place a satellite in orbit (Sputnik), and with this, the space era started. The objective of building rockets was not anymore only military, but it was also scientific and even commercial. 

	\item {\bf In your own words give a definition for a rocket mission.}\\
	
A mission is the objective that the government of a country or a private organization has for which a rocket is needed to be accomplished. Examples are; putting a satellite in GEO, sending a robot to Mars to make experiments, sending a missile with an atomic bomb to an asteroid that is going to impact our beautiful home planet.
	
	\item {\bf What is a payload?}\\
	
The payload is the equipment that solves the problem represented by the mission. In other words, the payload is how the mission is intended to be accomplished.

For the mission \lq\lq{}putting a satellite in GEO\rq\rq{}, the payload is the satellite.\\
For the mission \lq\lq{}sending a robot to Mars to make experiments\rq\rq{}, the payload is the robot.\\
For the mission \lq\lq{}sending a missile with an atomic bomb to an asteroid that is going to impact our beautiful home planet\rq\rq{}, the payload is the bomb.

The payload is the all hardware above the launch vehicle, except the protective fairing, which is also part of the launch vehicle. 
	
	\item {\bf What is the so-called “SMAD”?}\\
	
SMAD stands for \textit{Space Mission Analysis and Preparation}, and it is the name of book, that according to \cite{book}, is the standard text book for space mission preparation.
	
	\item {\bf Give the four basic assumptions required for understanding the basics of projectile motion.}\\
	
\begin{enumerate}
1. Acceleration due to gravity is constant\\
2. No air resistance\\
3. Earth is flat\\
4. Earth’s rotation has no impact on the motion of the projectile.
\end{enumerate}
	
	\item {\bf Define MECO.}\\
	
MECO stands for main engine cutoff, and it happens when the main engine is turned off, it can be because all the propellant has been used, or also in a multistage rocket it can be that the first stage that contains the main engine still has propellant which is going to be used for the landing of the first stage.
	
	\item {\bf Equation 2.9 gives the parabolic flight path of a rocket trajectory as height, \textit{y}, as a function of range, \textit{x}, or \textit{y(x)}. Use the quadratic equation to solve for \textit{x} as a function of \textit{y} to give a range equation as a function of height.}\\

\begin{equation} \label{eq:x_of_y}
	x(y-y_{bo})  = \frac{v_{bo}^2 \cos\theta}{g}(\sin\theta \pm \sqrt{\sin^2\theta-\frac{2g(y-y_{bo})}{v_{bo}^2}})+ x_{bo} 
\end{equation}

Figure \ref{fig:exercise7} shows \textit{x(y)}.
	
\begin{figure}[h!]
	\centering
	\includegraphics[width=1\textwidth]{exercise7.png}
	\caption{Range as a function of height \textit{x(y)}.}
	\label{fig:exercise7}
\end{figure}	
	
	\item {\bf A rocket is launched with a burnout velocity of 75 m/sec, burnout altitude of 300 m, and a burnout range of 100 m. Assuming a flight path angle of 75\degree, calculate the final range of the rocket when it impacts the ground.}\\

Substituting values into eq. 2.14 from the book. \\

\begin{equation}
	x_{max}  = x_{bo} + (v_{bo}\cos\theta)\left[ \frac{v_{bo}\sin\theta+\sqrt{v_{bo}^2\sin^2\theta+2gy_{bo}}}{g}\right]
\end{equation}

\begin{equation}
	x_{max}  =  100 + (75\cos75\degree)\left[ \frac{75\sin75\degree+\sqrt{75^2\sin^2(75\degree)+2(9.81)(300)}}{9.81}\right] = 452.1426 [m] 
\end{equation}

This value can also be observed in figure \ref{fig:exercises_8_9_10}.
	
	\item {\bf Calculate the maximum altitude reached by the rocket in Exercise 8.}\\
	
Substituting values into eq. 2.11 from the book. \\

\begin{equation}
	y_{max}  =  y_{bo} + \frac{v_{bo}^2 \sin^2\theta}{2g}
\end{equation}

\begin{equation}
	y_{max}  =  300 + \frac{75^2 \sin^2 75\degree}{2(9.81)} = 567.49 [m]
\end{equation}	

This value can also be observed in figure \ref{fig:exercises_8_9_10}.

	\item {\bf Redo Exercise 8 to determine the range at MECO altitude. What is the range at MECO if the initial flight path angle is 15\degree ?}\\
	
As explained in the book, the range at MECO for two angles whose addition equals 90\degree (75\degree + 15\degree = 90 \degree), the range at MECO is the same. Substituting values into equation \ref{eq:x_of_y} the range at MECO for 75\degree and 15\degree can be obtained.

\begin{equation}
	x(y_{MECO}-300)  = \frac{75^2 \cos15\degree}{9.81}(\sin15\degree + \sqrt{\sin^2 15\degree -\frac{2(9.81)(300-300)}{75^2}})+ 100 = 386.697 [m]
\end{equation}	
	

\begin{figure}[h!]
	\centering
	\includegraphics[width=1\textwidth]{exercises_8_9_10.png}
	\caption{Trajectory of rocket launched at 75\degree and 15\degree .}
	\label{fig:exercises_8_9_10}
\end{figure}


	
\end{enumerate}




























\section{Exercises of Chapter 2}

\begin{enumerate}

	\item {\bf Discuss the dichotomy of rocket science in the modern era.}\\
	
The first dichotomy starts after WWI, with the Treaty of Versailles that prohibited Germany to develop long-range artillery and the NAZI Germany developed the V2 (\textit{Vergeltungswaffe 2}), that was a missile capable of reaching an objective 300 km away from the launch place. The ''success'' of this lethal weapon, led the governments of the Soviet Union and USA to start developing their own missiles.

The second dichotomy happens when the Soviets where the firsts to place a satellite in orbit (Sputnik), and with this, the space era started. The objective of building rockets was not anymore only military, but it was also scientific and even commercial. 

	\item {\bf In your own words give a definition for a rocket mission.}\\
	
A mission is the objective that the government of a country or a private organization has for which a rocket is needed to be accomplished. Examples are; putting a satellite in GEO, sending a robot to Mars to make experiments, sending a missile with an atomic bomb to an asteroid that is going to impact our beautiful home planet.
	
	\item {\bf What is a payload?}\\
	
The payload is the equipment that solves the problem represented by the mission. In other words, the payload is how the mission is intended to be accomplished.

For the mission \lq\lq{}putting a satellite in GEO\rq\rq{}, the payload is the satellite.\\
For the mission \lq\lq{}sending a robot to Mars to make experiments\rq\rq{}, the payload is the robot.\\
For the mission \lq\lq{}sending a missile with an atomic bomb to an asteroid that is going to impact our beautiful home planet\rq\rq{}, the payload is the bomb.

The payload is the all hardware above the launch vehicle, except the protective fairing, which is also part of the launch vehicle. 
	
	\item {\bf What is the so-called “SMAD”?}\\
	
SMAD stands for \textit{Space Mission Analysis and Preparation}, and it is the name of book, that according to \cite{book}, is the standard text book for space mission preparation.
	
	\item {\bf Give the four basic assumptions required for understanding the basics of projectile motion.}\\
	
\begin{enumerate}
1. Acceleration due to gravity is constant\\
2. No air resistance\\
3. Earth is flat\\
4. Earth’s rotation has no impact on the motion of the projectile.
\end{enumerate}
	
	\item {\bf Define MECO.}\\
	
MECO stands for main engine cutoff, and it happens when the main engine is turned off, it can be because all the propellant has been used, or also in a multistage rocket it can be that the first stage that contains the main engine still has propellant which is going to be used for the landing of the first stage.
	
	\item {\bf Equation 2.9 gives the parabolic flight path of a rocket trajectory as height, \textit{y}, as a function of range, \textit{x}, or \textit{y(x)}. Use the quadratic equation to solve for \textit{x} as a function of \textit{y} to give a range equation as a function of height.}\\
	
	\item {\bf A rocket is launched with a burnout velocity of 75 m/sec, burnout altitude of 300 m, and a burnout range of 100 m. Assuming a flight path angle of 75\degree, calculate the final range of the rocket when it impacts the ground.}\\
	
	\item {\bf Calculate the maximum altitude reached by the rocket in Exercise 8.}\\

	\item {\bf Redo Exercise 8 to determine the range at MECO altitude. What is the range at MECO if the initial flight path angle is 15\degree ?}\\
	
\end{enumerate}

\cleardoublepage
%%%%%%%%%%%%%%%%%%%%%%%%%%%%%
%					References											 %
%%%%%%%%%%%%%%%%%%%%%%%%%%%%%

\begin{thebibliography}{99}

\bibitem{book}
	Travis S. Taylor,
	\emph{Introduction to Rocket Science and Engineering}.
	CRC Press,
	2009.
	
\bibitem{aeolipile}
	Aeolipile\\
	Wikipedia, the free encyclopedia\\
	April 2017\\
	\url{https://en.wikipedia.org/wiki/Aeolipile}
	
\bibitem{gunpowder}
	Gunpowder\\
	Wikipedia, the free encyclopedia\\
	April 2017\\
	\url{https://en.wikipedia.org/wiki/Gunpowder}

\bibitem{principia}
	\textit{Philosophiæ Naturalis Principia Mathematica}\\
	Wikipedia, the free encyclopedia\\
	April 2017\\
	\url{https://en.wikipedia.org/wiki/Philosophiae_Naturalis_Principia_Mathematica}
	
\bibitem{konstantin}
	Konstantin Tsiolkovsky\\
	Wikipedia, the free encyclopedia\\
	April 2017\\
	\url{https://en.wikipedia.org/wiki/Konstantin_Tsiolkovsky}
	
\bibitem{goddard}
	Robert H. Goddard\\
	Wikipedia, the free encyclopedia\\
	April 2017\\
	\url{https://en.wikipedia.org/wiki/Robert_H._Goddard}	
	
\bibitem{korolev}	
	Sergei Korolev\\
	Wikipedia, the free encyclopedia\\
	April 2017\\
	\url{https://en.wikipedia.org/wiki/Sergei_Korolev}	

\bibitem{vonbraun}	
	Wernher von Braun\\
	Wikipedia, the free encyclopedia\\
	April 2017\\
	\url{https://en.wikipedia.org/wiki/Wernher_von_Braun}

\bibitem{vanguard1}
	Vanguard 1\\
	Wikipedia, the free encyclopedia\\
	April 2017\\
	\url{https://en.wikipedia.org/wiki/Vanguard_1}
	
\bibitem{udmh}
	Unsymmetrical dimethylhydrazine\\
	Wikipedia, the free encyclopedia\\
	April 2017\\
	\url{https://en.wikipedia.org/wiki/Unsymmetrical_dimethylhydrazine}
	
\bibitem{nto}
	Dinitrogen tetroxide\\
	Wikipedia, the free encyclopedia\\
	April 2017\\
	\url{https://en.wikipedia.org/wiki/Dinitrogen_tetroxide}



\end{thebibliography}

\end{document}
