% This document is under construction

\documentclass{article}
\usepackage[utf8]{inputenc}
\usepackage{hyperref}
\usepackage{graphicx}
\usepackage{multirow}
\usepackage{mhchem}

\begin{document}

\title{Solution to the exercises of the book ``Introduction to Rocket Science and Engineering''}
\author{Arturo Gonzalez\\
	\texttt{\href{mailto:arturo.gonzalez@argonur.com}{arturo.gonzalez@argonur.com}}}
\date{\today}
\maketitle

\begin{abstract}
This document contains the solution to the exercises of each chapter of the book ``Introduction to Rocket Science and Engineering'' by Travis S. Taylor. Solving the exercises is a project proposed in the blog \url{https://www.argonur.com}.\\
To visit the website of the project: \\
\url{https://argonur.com/solucion-introduction-to-rocket-science-and-engineering/}. 
\end{abstract}

\cleardoublepage

%%%%%%%%%%%%%%%%%%%%%%%%%%%%%
%					Chapter 1											 %
%%%%%%%%%%%%%%%%%%%%%%%%%%%%%

%\section{Exercises of Chapter 1}

\begin{enumerate}
	\item {\bf Discuss the relevance of the \textit{aelopile} to rocket science and why it was considered the first demonstration of the principle of rocketry.}\\

	\item {\bf What are the main components of the gun powder?}\\
	
	\item {\bf What was \textit{Principia} and why is it relevant to rocket science?}\\
	
	\item {\bf Why were William Hale’s rockets “better” than William Congreve’s?}\\
	
	\item {\bf Compare and contrast the contributions to the development of rocketry by Konstantin Tsiolkovsky and Robert Goddard. Which one could be considered the “father of rocket science” and which one the “father of rocket engineering”?}\\
	
	\item {\bf Who was known as the Chief Designer and why?}\\
	
	\item {\bf Who was the Chief Designer’s counterpart in the American space program?}\\
	
	\item {\bf What is the oldest spacecraft still in orbit?}\\
	
	\item {\bf What is UDMH? What is it used for? What is NTO?}\\
	
	\item {\bf Draw a simple liquid fuel rocket and label all the major subcomponents.}\\
	
\end{enumerate}

\section{Exercises of Chapter 1}

\begin{enumerate}
	\item {\bf Discuss the relevance of the \textit{aelopile} to rocket science and why it was considered the first demonstration of the principle of rocketry.}\\

The aeolipile consists of a vessel, usually a "simple" solid of revolution, such as a sphere or a cylinder, arranged to rotate on its axis, having oppositely bent or curved nozzles projecting from it (tipjets). When the vessel is pressurised with steam, steam is expelled through the nozzles, which generates thrust due to the rocket principle as a consequence of the 2nd and 3rd of Newton's laws of motion. When the nozzles, pointing in different directions, produce forces along different lines of action perpendicular to the axis of the bearings, the thrusts combine to result in a rotational moment (mechanical couple), or torque, causing the vessel to spin about its axis. \cite{aeolipile}

	\item {\bf What are the main components of the gunpowder?}\\
	
Gunpowder, also known as black powder, is the earliest known chemical explosive. It is a mixture of sulfur, charcoal, and potassium nitrate (saltpeter). The sulfur and charcoal act as fuels, and the saltpeter is an oxidizer. Because of its burning properties and the amount of heat and gas volume that it generates, gunpowder has been widely used as a propellant in firearms, as a pyrotechnic composition in fireworks and as a blasting powder in quarrying, mining, and road building. \cite{gunpowder}
	
	\item {\bf What was \textit{Principia} and why is it relevant to rocket science?}\\
	
	\item {\bf Why were William Hale’s rockets “better” than William Congreve’s?}\\
	
	\item {\bf Compare and contrast the contributions to the development of rocketry by Konstantin Tsiolkovsky and Robert Goddard. Which one could be considered the “father of rocket science” and which one the “father of rocket engineering”?}\\
	
	\item {\bf Who was known as the Chief Designer and why?}\\
	
	\item {\bf Who was the Chief Designer’s counterpart in the American space program?}\\
	
	\item {\bf What is the oldest spacecraft still in orbit?}\\
	
Vanguard 1 (ID: 1958-Beta 2) was the fourth artificial Earth orbital satellite launched (after Sputnik 1, Sputnik 2, and Explorer 1). It was the first satellite to be solar powered. Although communication with it was lost in 1964, it remains the oldest manmade satellite still in orbit. It was designed to test the launch capabilities of a three-stage launch vehicle as a part of Project Vanguard, and the effects of the environment on a satellite and its systems in Earth orbit. It also was used to obtain geodetic measurements through orbit analysis. Vanguard 1 was described by then-Soviet Premier Nikita Khrushchev as, "The grapefruit satellite." \cite{vanguard1}
	
	\item {{\bf What is UDMH? What is it used for? What is NTO?}}\\

{\bf UDMH}\\
Unsymmetrical dimethylhydrazine (UDMH; 1,1-dimethylhydrazine) is a chemical compound with the formula \ce{H2NN(CH3)2}. It is a colourless liquid, with a sharp, fishy, ammoniacal smell typical for organic amines. Samples turn yellowish on exposure to air and absorb oxygen and carbon dioxide. It mixes completely with water, ethanol, and kerosene. In concentration between 2.5\% and 95\% in air, its vapors are flammable. It is not sensitive to shock.

UDMH is often used in hypergolic rocket fuels as a bipropellant in combination with the oxidizer nitrogen tetroxide and less frequently with IRFNA (red fuming nitric acid) or liquid oxygen. UDMH is a derivative of hydrazine and is sometimes referred to as a hydrazine.

UDMH is often used in hypergolic rocket fuels as a bipropellant in combination with the oxidizer nitrogen tetroxide and less frequently with IRFNA (red fuming nitric acid) or liquid oxygen. UDMH is a derivative of hydrazine and is sometimes referred to as a hydrazine. \cite{udmh}

{\bf NTO}\\
Dinitrogen tetroxide, commonly referred to as nitrogen tetroxide, is the chemical compound \ce{N2O4}. It is a useful reagent in chemical synthesis. It forms an equilibrium mixture with nitrogen dioxide.

Dinitrogen tetroxide is a powerful oxidizer that is hypergolic (spontaneously reacts) upon contact with various forms of hydrazine, which makes the pair a popular bipropellant for rockets.

Nitrogen tetroxide is used as an oxidizer in one of the more important rocket propellants because it can be stored as a liquid at room temperature. In early 1944 research on the usability of dinitrogen tetroxide as an oxidizing agent for rocket fuel was conducted by German scientists, although the Nazis only used it to a very limited extent as an additive for S-Stoff (fuming nitric acid). It became the storable oxidizer of choice for many rockets in both the United States and USSR by the late 1950s. It is a hypergolic propellant in combination with a hydrazine-based rocket fuel. One of the earliest uses of this combination was on the Titan family of rockets used originally as ICBMs and then as launch vehicles for many spacecraft. Used on the U.S. Gemini and Apollo spacecraft and also on the Space Shuttle, it continues to be used as stationkeeping propellant on most geo-stationary satellites, and many deep-space probes. It now seems likely that NASA will continue to use this oxidizer in the next-generation 'crew-vehicles' which will replace the shuttle. It is also the primary oxidizer for Russia's Proton rocket.

When used as a propellant, dinitrogen tetroxide is usually referred to simply as 'Nitrogen Tetroxide' and the abbreviation 'NTO' is extensively used. Additionally, NTO is often used with the addition of a small percentage of nitric oxide, which inhibits stress-corrosion cracking of titanium alloys, and in this form, propellant-grade NTO is referred to as "Mixed Oxides of Nitrogen" or "MON". \cite{nto}
	
	\item {\bf Draw a simple liquid fuel rocket and label all the major subcomponents.}\\
	
\end{enumerate}

%%%%%%%%%%%%%%%%%%%%%%%%%%%%%
%					End Chapter 1										 %
%%%%%%%%%%%%%%%%%%%%%%%%%%%%%

\cleardoublepage

\begin{thebibliography}{99}

\bibitem{book}
	Travis S. Taylor,
	\emph{Introduction to Rocket Science and Engineering}.
	CRC Press,
	2009.
	
\bibitem{aeolipile}
	Aeolipile\\
	Wikipedia, the free encyclopedia\\
	April 2017\\
	\url{https://en.wikipedia.org/wiki/Aeolipile}

\bibitem{gunpowder}
	Gunpowder\\
	Wikipedia, the free encyclopedia\\
	April 2017\\
	\url{https://en.wikipedia.org/wiki/Gunpowder}
	
\bibitem{vanguard1}
	Vanguard 1\\
	Wikipedia, the free encyclopedia\\
	April 2017\\
	\url{https://en.wikipedia.org/wiki/Vanguard_1}
	
\bibitem{udmh}
	Unsymmetrical dimethylhydrazine\\
	Wikipedia, the free encyclopedia\\
	April 2017\\
	\url{https://en.wikipedia.org/wiki/Unsymmetrical_dimethylhydrazine}
	
\bibitem{nto}
	Dinitrogen tetroxide\\
	Wikipedia, the free encyclopedia\\
	April 2017\\
	\url{https://en.wikipedia.org/wiki/Dinitrogen_tetroxide}	

\end{thebibliography}

\end{document}
